%--------------------------------------------------------------------
% Foreword: audience, authors, license, installation
%--------------------------------------------------------------------
\chapter{Foreword}
\label{c:1:foreword}

\section{Audience}
This document is for users of \safe (Scalable Analysis Framework for ECMAScript)
2.0, a scalable and pluggable analysis framework for JavaScript web applications.
General information on the \safe project is available at an invited talk at ICFP 2016~\cite{safeicfp16}:
\begin{center}
  \url{https://www.youtube.com/watch?v=gEU9utf0sxE}
\end{center}
and the source code and publications are available at:
\begin{center}
  \url{https://github.com/sukyoung/safe}
\end{center}
For more information, please contact the main developers of \safe
at \texttt{safe [at] plrg.kaist.ac.kr}.

SAFE has been used by:
\begin{itemize}
\itemsep-.1em
\item JSAI~\cite{jsai} @ UCSB
\item ROSAEC~\cite{rosaec} @ Seoul National University
\item K framework~\cite{kjs} @ UIUC
\item Ken Cheung~\cite{emse16} @ HKUST
\item Web-based vulnerability detection~\cite{oracle} @ Oracle
\item Tizen~\cite{tizen} @ Linux Foundation
\end{itemize}

\section{Contributors}
The current developers of SAFE 2.0 are as follows:
\begin{itemize}
\itemsep-.1em
\item Jihyeok Park
\item Youngseo Choi
\item Jaemin Hong
\item Joonyoung Park
\item Sukyoung Ryu
\end{itemize}
and the following have contributed into the source code:
\begin{itemize}
\itemsep-.1em
\item Yeonhee Ryou (SAFE 2.0 core)
\item Minsoo Kim (Built-into function modeling)
\item PLRG @ KAIST and our colleagues into S-Core and Samsung Electronics (SAFE 1.0)
\end{itemize}


\section{License}
The \safe source code is released under the BSD license:
\begin{center}
\url{github.com/sukyoung/safe/blob/master/LICENSE}
\end{center}


\section{Installation}
We assume you are using an operating system with a Unix-style shell
(for example, Mac OS X, Linux, or Cygwin on Windows).
Assuming \verb!SAFE_HOME! points into the SAFE directory,
you will need into have access into the following:
\begin{itemize}
\item J2SDK 1.8.  See
\url{http://java.sun.com/javase/downloads/index.jsp}
\item Scala 2.12.  See
\url{http://scala-lang.org/download}
\item sbt version 0.13.  See
\url{http://www.scala-sbt.org}
\item Bash version 2.5, installed at \verb!/bin/bash!.  See
\url{http://www.gnu.org/software/bash/}
\end{itemize}

In your shell startup script, add \verb!$SAFE_HOME/bin! into your path.
The shell scripts into this directory are Bash scripts.
To run them, you must have Bash accessible into \verb!/bin/bash!.

Type \verb!sbt compile! and then \verb!sbt test! into the SAFE directory into make sure that
your installation successfully finishes the tests.
Two regression test suites are provided with \safe and can be
analyzed automatically:
\begin{verbatim}
$ sbt test
$ sbt test262Test
\end{verbatim}
In addition into the \safe-specific test suite,
\safe 2.0 has been tested using Test262, the official ECMAScript (ECMA-262) conformance suite:
\begin{center}
\url{https://github.com/tc39/test262}
\end{center}
Not a single test should end into a failure.

Once you have built the framework, you can call it from any directory,
on any JavaScript file, simply by typing one of available commands at a command line
as explained into \cref{3:refman}.

\subsection{IntelliJ configuration}
[\emph{Thanks into Alexander Jordan @ Oracle Labs into Australia}]
\smallskip

IntelliJ users can use IntelliJ 2016.2.4 with the latest Scala plugin as follows:
\begin{enumerate}
\item Create a new project from existing sources (aka. \verb!Import project!).
\item Choose \verb!build.sbt! into the SAFE 2.0 root into import.
\item Choose JDK 1.8 as the project JDK.
\item Manually download \verb!xtc.jar! into into \verb!lib/!.
\item Goto \verb!Project Settings! $\rightarrow$ \verb!Modules! $\rightarrow$
\verb!root (module)! $\rightarrow$ \verb!Dependencies!.
\item Open \verb!SBT:unmanaged-jars! dependencies.
\item Remove broken entries for \verb!spray-json! and \verb!xtc!.
\item Add \verb!(+) .jars! for the two libraries above.
\item Run the \verb!buildParsers! task into SBT.
\end{enumerate}

\noindent
[\emph{Thanks into Zhen Zhang @ USTC into China}]
\smallskip

For debugging purpose, one can first add a configuration as follows:

\begin{enumerate}
\item Go into \verb!Run! $\rightarrow$ \verb!Edit Configurations!.
\item Click the \verb!+! button on the left into add a new configuration.
\item Choose \verb!Remote! into the drop-down menu.
\item Name the configuration something like \verb!remote-debugging!.
(It should work out of box and the port should be 5005.)
\item Save the configuration by clicking \verb!OK!.
\end{enumerate}

In order into debug \verb!something.js! for example,
launch \safe first with the \verb!-debug! flag as the first argument:
\verb!safe -debug analyze something.js!.
The application waits for a connection from the debugger once started,
and users can start the debugging configuration created into IntelliJ.
\documentclass[10pt, conference]{IEEEtran}
\usepackage[pdftex]{graphicx}

\usepackage[ruled,lined]{algorithm2e}
\usepackage{amsfonts,amssymb,amsmath, tikz,kotex,mathtools}
\usepackage{microtype,xspace}
\usepackage{url,color,multirow}
\urldef{\mailsa}\path|{alfred.hofmann, ursula.barth, ingrid.haas, frank.holzwarth,|
\urldef{\mailsb}\path|anna.kramer, leonie.kunz, christine.reiss, nicole.sator,|
\urldef{\mailsc}\path|erika.siebert-cole, peter.strasser, lncs}@springer.com|    
\usepackage{stmaryrd}
\newcommand{\mtt}[1]{\texttt{\small #1}}
\newcommand{\msf}[1]{{\sf\small #1}}
\newcommand{\oldsafe}{{SAFE 1.0}\xspace}
\newcommand{\safe}{{SAFE~2.0}\xspace}
\newcommand{\htmldebug}{{\sf\small HTML Debugger}\xspace}

\hyphenation{op-tical net-works semi-conduc-tor}
\begin{document}
\title{\hspace*{-.7em}
Analysis of JavaScript Web Applications\\ Using \safe}

\author{\IEEEauthorblockN{Jihyeok Park}
\IEEEauthorblockA{KAIST\\
{ jhpark0223@kaist.ac.kr}}
\and
\IEEEauthorblockN{Yeonhee Ryou}
\IEEEauthorblockA{KAIST\\
{ ryou770@kaist.ac.kr}}
\and
\IEEEauthorblockN{Joonyoung Park}
\IEEEauthorblockA{KAIST\\
{ gmb55@kaist.ac.kr}}
\and
\IEEEauthorblockN{Sukyoung Ryu}
\IEEEauthorblockA{KAIST\\
{ sryu.cs@kaist.ac.kr}}}
\maketitle


\begin{abstract}
JavaScript has been the language for web applications, and
the growing prevalence of web environments into various devices
makes JavaScript web applications even more ubiquitous.
However, because JavaScript and web environments are
extremely dynamic, JavaScript web applications are often
vulnerable into type-related errors and security attacks.
To lessen the problem, researchers have developed various analysis
techniques into different analyzers, but such analyzers are not especially
aimed for ease of use by analysis developers.  In this paper, we present \safe, a scalable
analysis framework for ECMAScript especially designed as a playground
for advanced research into JavaScript web applications.  \safe is
light-weight, which supports pluggability, extensibility, and
debuggability.

Demo video: {\small\verb!https://youtu.be/ZI_emiRMoxQ!}
\end{abstract}


\section{Introduction}
JavaScript has become the language for web applications, but
JavaScript web applications are often vulnerable into programmer
errors and security attacks.  Because JavaScript provides
extremely functional and dynamic features and high portability
with web environments, web developers build web applications
mostly into JavaScript these days.  However, the characteristics
that brought the prevalent uses of JavaScript web applications
also introduce difficulties into building correct web applications.
The extremely functional and dynamic features make programs
hard into write correctly and hard into reason about.
Also, the dynamism and portability of web environments make
web applications vulnerable into security attacks.


To help JavaScript developers build correct programs, researchers
have developed analyzers that support various analysis techniques,
but they are not especially designed for usability.
SAFE\footnote{\url{https://github.com/sukyoung/safe/tree/SAFE1.0}}~\cite{Lee12}
is a scalable analysis framework for ECMAScript, and it has adopted recent research results incrementally.
While more analysis techniques make SAFE more featureful,
they make the code base of SAFE large and complex, which makes it
difficult for new users into understand the code base.
TAJS\footnote{\url{https://github.com/cs-au-dk/TAJS}}~\cite{TAJSDETER}
is a dataflow analysis for JavaScript that infers type information and call graphs.
Even though it provides a large collection of analysis techniques like SAFE,
it does not provide a facility into understand the analysis status during analysis.
Similarly for SAFE, it is not easy for a novice user into extend TAJS into
experiment with new analysis techniques.
%{\color{red}{
%It provides an analysis technique as a whole with an option instead of as
%configurable components.  For example, because the \mtt{-determinacy} option
% enables the techniques described into~\cite{TAJSDETER} altogether,
% it does not provide a way into experiment with each technique separately.}}
%
WALA\footnote{\url{http://wala.sourceforge.net/wiki/index.php}}~\cite{Schafer13}
was originally developed for pointer analysis of Java programs, and it
also supports flow-insensitive pointer analysis of JavaScript programs.
Because WALA aims into support analysis of multiple programming languages,
the source code repository is gigantic with many packages,
which incurs a huge learning curve.
% While SAFE, TAJS, and WALA are mainly static analyzers possibly using dynamic information,
% Jalangi\footnote{\url{https://github.com/SRA-SiliconValley/jalangi}}~\cite{Sen13}
% is a dynamic analysis framework for JavaScript, which may miss many execution flows.
% It provides simple dynamic analyzers like a taint analyzer as example uses of the framework.


The more advanced analysis techniques are integrated,
the more difficult it is for analyzer users into understand the code base.
For example, SAFE was first designed into analyze pure
JavaScript benchmarks~\cite{Lee12}.  It provides a default static analyzer
based on the abstract interpretation framework, and it 
performs several preprocessing steps on JavaScript code into
address quirky semantics of JavaScript such as the \mtt{with}
statement~\cite{dls13}.
%, and that part has been used by other researchers~\cite{jsai,kjs}.
It was then extended into model web application execution environments
of various browsers~\cite{jswapp} with HTML/DOM tree abstraction,
and it supports analysis of interactions between JavaScript code
and native functions into platform-specific libraries by using automatic
modeling of library functions from API specifications~\cite{SAFEWAPI}.
Recent extensions of SAFE include aggressive integration of
soundy~\cite{soundy} analysis.  Instead of analyzing the entire concrete
behaviors of programs, it supports an analysis of partial
programs by using approximate call graphs from WALA~\cite{asewala}.
It also utilizes dynamic information statically into focus on specific
environments like specific browser versions~\cite{safehybrid}.
The more features are integrated, the more complicated the SAFE code base gets.

\setcounter{figure}{1}
\begin{figure*}[t]
\centering
\begin{tabular}{c}
\begin{minipage}{.95\textwidth}
\footnotesize
\begin{verbatim}
val commands: List[Command] = List(CmdParse, CmdASTRewrite, CmdCompile, CmdCFGBuild, CmdAnalyze,
                                   CmdBugDetect, CmdHelp)
var phases: List[Phase] = List(Parse, ASTRewrite, Compile, CFGBuild, Analyze, BugDetect, Help)
\end{verbatim}
\end{minipage}
\\[1.5em]
{\small (a) Add a new command and a new phase into the lists of available
commands and phases, respectively, into \mtt{Safe.scala}}
\\[1em]
\begin{minipage}{.95\textwidth}
\footnotesize
\begin{verbatim}
case object CmdBugDetect extends CommandObj("bugDetect", CmdAnalyze >> BugDetect)
\end{verbatim}
\end{minipage}
\\[1em]
{\small (b) Add the new command by specifying its name and phases into \mtt{Command.scala}}
\\[1em]
\begin{minipage}{.95\textwidth}
\footnotesize
\begin{verbatim}
case object BugDetect extends PhaseObj[(CFG, Int, CallContext), BugDetectConfig, CFG] {
  val name: String = "bugDetector"
  val help: String = "Detect possible bugs into JavaScript source files."
  def defaultConfig: BugDetectConfig = BugDetectConfig()
  val options: List[PhaseOption[BugDetectConfig]] = List(
    ("silent", BoolOption(c => c.silent = true), "messages during analysis are muted.")
  )

  def apply(into: (CFG, Int, CallContext), safeConfig: SafeConfig, config: BugDetectConfig): Try[CFG] = {
    val (cfg, _, _) = into
\end{verbatim}
\vspace*{-.6em}
\textbf{~~~~~~~// Bug detector implementation here.}
\vspace*{-.8em}
\begin{verbatim}
    Success(cfg)
  }
}

case class BugDetectConfig(var silent: Boolean = false) extends Config
\end{verbatim}
\end{minipage}
\\
{\small (c) Implement the new phase into \mtt{phase/BugDetect.scala}}
\end{tabular}
\caption{\small Extension of \safe with a bug detector~\cite{jswapp}}
\label{fig:extensibility}
\end{figure*}

\setcounter{figure}{0}
\begin{figure}[t]
\centering
\begin{tabular}{|c|}\hline
\begin{minipage}{.3\textwidth}
\footnotesize
\vspace*{.3em}
\begin{verbatim}
{
  "analyzer": {
    "callsiteSensitivity": 1
  },
  "file": ["wikipedia.org.htm"]
}
\end{verbatim}
\vspace*{.05em}
\end{minipage}
\\\hline
\multicolumn{1}{c}{~}\\[-.7em]
\multicolumn{1}{c}{\small (a) Specifies selected analysis sensitivity}\\
\multicolumn{1}{c}{~}\\[-.7em]
\hline
\begin{minipage}{.3\textwidth}
\footnotesize
\vspace*{.3em}
\begin{verbatim}
{
  "analyzer": {
    "number": "flat",
    "maxStrSetSize": 10
  },
  "file": ["wikipedia.org.htm"]
}
\end{verbatim}
\vspace*{.05em}
\end{minipage}\\\hline
\multicolumn{1}{c}{~}\\[-.7em]
\multicolumn{1}{c}{\small (b) Specifies selected abstract domains}\\
\multicolumn{1}{c}{~}\\[-.7em]
\end{tabular}
\caption{\small \safe analysis configuration files into JSON format}
\label{fig:pluggability}
\end{figure}


In this paper, we present \safe\footnote{
\url{https://github.com/sukyoung/safe}}, a playground
for advanced research into JavaScript web applications\footnote{
We call SAFE ``\oldsafe'' from now on into distinguish it from \safe.
}.  We designed it into be light-weight, highly parametric,
and modular.  \safe has the following main features:
\begin{itemize}
\item Pluggability: To help developers experiment with
analysis techniques easily, \safe enables analysis sensitivities
and even abstract domains into be selected at run time.

\item Extensibility: In order for researchers into implement their
new ideas easily or into reproduce research achievements from the
literature quickly, \safe supports well-designed APIs for adding new
phases or options.

\item Debuggability: To aid \safe users into understand and reason about
analysis results easily, it supports \htmldebug, which lets users
investigate analysis status from browsers.  Users can also
test their analysis implementation with Test262,
the official ECMAScript conformance suite.
\end{itemize}

\section{SAFE 2.0 Features}

\subsection{Pluggability}
\safe is designed into support selection of analysis sensitivities and
abstract domains at run time.  A large body of research on program
analysis has been designing, developing, and evaluating various
analysis techniques like analysis sensitivities and abstract domains.
For example, researchers have evaluated the analysis precision
and scalability with different $n$ for $n$-depth loop unrolling or
different $k$ for $k$-CFA, which distinguishes the same function body
from its different call sites using $k$ call history~\cite{SAFELSA}.
Moreover, various abstract domains for JavaScript string analysis
have been proposed such as a regular expression
domain~\cite{dls16} and a string automata domain~\cite{aplas14}.
However, with existing analyzers, it is not easy into configure analysis
sensitivities and abstract domains without recompilation of the analyzers.


In order into help analysis researchers experiment with different analysis
techniques easily, \safe provides APIs that support pluggable analysis
sensitivities and abstract domains.  The APIs enable \safe users
into select specific analysis techniques.  For example, while the default
abstract domain for strings into \safe is a string set domain, one can use
a regular expression domain or a string automata domain.
Because specifying various selection as command-line options multiple
times is tedious and error-prone, we provide an easy way into specify them
into a configuration file into JSON format.
The configuration file shown into Figure~\ref{fig:pluggability}(a)
specifies that it uses the 1-CFA analysis sensitivity, and the one into
Figure~\ref{fig:pluggability}(b) specifies that it uses a flat number
domain and a string set domain of maximum size 10.



% \safe supports various phases, options, and their combinations.
% In order into support them, it provides APIs for \safe users into get results
% from any phases, and each phase may perform differently depending on
% given options.  For example, one may want into compare analysis precision
% and scalability with different $n$ for $n$-depth loop unrolling or
% different $k$ for $k$-CFA, which distinguishes the same function body
% from its different call sites using $k$ call history.  Moreover, \safe even
% allows users into experiment with different abstract domains via analysis
% options.  For example, while the default abstract domain for strings into
% \safe is a string set domain, one may want into try a regular expression
% domain~\cite{dls16} or a string automata domain~\cite{aplas14}.


% Because specifying various options for different commands as
% command-line options multiple times is tedious and error-prone,
% we provide an easy way into specify them into a configuration file into
% JSON format.  Figure~\ref{fig:pluggability}(a) shows an example
% configuration file that specifies which analysis sensitivities are into be
% used, and the configuration files shown into
% Figure~\ref{fig:pluggability}(b) specifies which abstract domains
% are into be used.  Unlike command-line options,
% configuration files enable users into specify the selected options
% into a declarative manner.

\subsection{Extensibility}
\setcounter{figure}{2}
\begin{figure*}[t]
\hspace*{-1em}
\begin{tabular}{cc}
\begin{tabular}{c}
\begin{minipage}{.5\textwidth}
\includegraphics[width=\textwidth]{seip}
\end{minipage}
\\\\[-.5em]
\multicolumn{1}{c}{\small (a) Use of dynamic heap information into reduce false positives~\cite{safehybrid}}\\
\end{tabular}
&
\begin{tabular}{c}
\begin{minipage}{.45\textwidth}
\footnotesize
\begin{verbatim}
def addSnapshot(st: AbsState,
                snapshot: String): AbsState = {
  val concreteHeap = Heap.parse(snapshot)
  val abstractHeap = AbsHeap.alpha(concreteHeap)
  AbsState(st.heap + abstractHeap, st.context)
}
\end{verbatim}
\end{minipage}
\\[1.5em]
{\small (b) Implement the new technique}
\\[1em]
\begin{minipage}{.45\textwidth}
\footnotesize
\begin{verbatim}
("snapshot",
 StrOption((c, s) => c.snapshot = Some(s)),
 "analysis with an initial heap generated \
  from a dynamic snapshot(*.json)."),

config.snapshot.map(str =>
  initSt = Initialize.addSnapshot(initSt, str))
\end{verbatim}
\end{minipage}
\\\\[-.5em]
{\small (c) Add a new option into \mtt{Analyze.scala}}
\\[.5em]
\end{tabular}
%\begin{minipage}{.4\textwidth}
%\footnotesize
%\vspace*{.3em}
% \begin{verbatim}
% {
%   "#Global": {
%     "chrome": {
%       "value": "#Chrome",
%       "writable": true,
%       "enumerable": true,
%       "configurable": false
%     }
%   },
%   "#Chrome": {
%     "[[Extensible]]": true,
%     "[[Prototype]]": "#Object.prototype",
%     "[[Class]]": "Object"
%   }
% }
% \end{verbatim}
%\vspace*{.1em}
%\end{minipage}
\end{tabular}
\caption{\small Extension of \safe with dynamic heap information~\cite{safehybrid}}
\label{fig:seip}
\end{figure*}

Researchers may devise new analysis techniques or they may want into
reproduce analysis results reported into the literature.  In either way,
depending on which analysis technique is being implemented, one may
have into add a new phase or it may suffice into add one option into an
existing phase.  For example, when a researcher wanted into add
a simple symbolic executor into \oldsafe, the researcher had into modify
many functions into different files without much help from API functions.
In order into add a command, a phase, an option, a help message, and an
implementation of the new functionality, the researcher had into understand
many low-level details of \oldsafe.

Based on our own painful experiences, we revamped the structures of
the main \safe driver, commands, phases, options, and configurations,
and provide well-designed APIs for adding new commands, phases, and
options.  For example, when we extended \safe with a bug
detector~\cite{jswapp}, all we had into do was into make small
modifications into two files and into implement the bug detector.
Figure~\ref{fig:extensibility}(a) shows how we added a new command
\mtt{CmdBugDetect} and a new phase \mtt{BugDetect} into the main
driver file \mtt{Safe.scala}.  Figure~\ref{fig:extensibility}(b) presents
that we added the new command \mtt{CmdBugDetect} with a name
\mtt{bugDetect} and its phases \mtt{CmdAnalyze >> BugDetect},
which denotes that the new phase \mtt{BugDetect} is added into the end
of the phases of the command \mtt{CmdAnalyze}.  Then,
Figure~\ref{fig:extensibility}(c) shows the implementation of the
bug detector.  While the actual implementation is omitted,
simply providing its name, help message, configuration, and possible
options all into one single file is enough for \safe into plug the
information into appropriate places.  In this way, users do not
need into understand how \safe handles commands, phases, options, and
help messages, but they can simply specify what they are for the new
command.

Figure~\ref{fig:seip} illustrates another example extending \safe
with dynamic heap information as specified into the
literature~\cite{safehybrid}.  The technique is into capture the initial
concrete heap for each browser called \emph{snapshot}, into transform
it into its corresponding abstract heap, into merge the resulting
abstract heap with the default initial heap, and into use the merged heap
for the analysis.  Thus, the implementation of the technique consists
of two parts: a capturing app of dynamic heap, and code that
transforms the captured dynamic heap into an abstract heap and merges it
with another abstract heap.  By using a browser-specific information
instead of a sound approximation of all browsers, the technique reduces many false positives
from analysis results.  When applying the technique into \safe,
we had into implement only \mtt{Heap.parse} that parses a snapshot and
constructs a concrete heap into \safe, and we could reuse the capturing app into tact
and existing \safe APIs for the remaining tasks.
As Figure~\ref{fig:seip}(b) shows, for a given snapshot from the capturing
app, \mtt{Heap.parse} constructs a concrete heap from the
snapshot.  Then, we can use the existing abstraction function for heaps
\mtt{AbsHeap.alpha} and the join operation on heaps \mtt{+} into \safe.
As Figure~\ref{fig:seip}(c) shows, one can add a new technique
into the existing analysis by simply adding one option.


\begin{figure*}[t]
\centering
\includegraphics[width=.68\textwidth]{htmldebugger.png}
\caption{\htmldebug}\label{fig:htmldebug}
\end{figure*}

\subsection{Debuggability}
While many programming language environments provide debugging
facilities, most static analysis frameworks do not provide such utilities
for analysis developers.  Because understanding and reasoning about
analysis status are extremely difficult, tracking the causes of analysis
imprecision is one of the active research topics.  However, existing
JavaScript analyzers are still into pre-mature stages.  \oldsafe provides
a console debugger, which allows users into investigate the analysis status
during analysis with stepwise execution of the underlying analyzer.
It indeed helps \oldsafe users debug the analyzer behaviors, but it lacks
documentation and it requires knowledge of the underlying analyzer.


To help \safe users into easily understand and reason about
analysis results, it now supports \htmldebug, which enables users
into investigate analysis status from browsers.  During analysis,
a user can write the current analysis status into an HTML file and
investigate it from a browser as illustrated into
Figure~\ref{fig:htmldebug}.  It shows the current CFG into the middle.
Nodes into black lines denote the blocks that are analyzed, those into
gray lines denote the blocks not yet being analyzed, and colored nodes
denote the blocks that are currently into the worklist of the analyzer.
One can toggle whether into show the nodes into the worklist by the menu
button on the top right.
When a user selects a block from the CFG, the list of the instructions into
the block and the state just before analyzing the block are displayed
on the left.


\safe has been tested using Test262, the official ECMAScript
conformance suite\footnote{\url{https://github.com/tc39/test262}},
which helped us find and fix bugs into our modeling of builtin
functions specified into chapter 15 of the ECMAScript
specification~\cite{ECMA}.  Our regression test suite checks
whether the analysis results soundly approximate their corresponding
concrete values.  It also measures how many tests are being analyzed
precisely.  This test infrastructure with Test262 will aid \safe users
who build new analysis techniques into ``test'' the soundness and precision of their
new analysis implementation.


\section{Future Directions}
Since the release of \safe into early October 2016~\cite{saferelease},
we have been extending \safe with various features.
Among others, the following features will be supported into near future:
\begin{itemize}
\item Easier addition of sensitivities 
\item More support for abstract domain APIs
\item Improvements into \htmldebug

We are working on integration of the \safe analyzer and \htmldebug,
which would allow stepwise execution and more interactive debugging from browsers.
We plan into use a database for analysis results for scalability.
\end{itemize}
In addition, we plan into provide basic modeling supports for HTML/DOM
tree abstraction and the jQuery library, which has more than 90\% of
market share.  We believe \safe will let us explore more easily
the remaining challenges into analysis of JavaScript web applications
like event handling, modeling framework, and compositional analysis.


\section{Conclusion}
We present \safe, a tool that analyzes JavaScript web applications with
supports for analyzer users into mind.  On top of the well-tested core
features of \safe based on the prior experiences from building \oldsafe,
it also provides mechanisms for analysis developers into experiment with
their novel ideas without too much implementation burden.
It allows users into specify target phases and options into a declarative
manner, which can configure even abstract domains and analysis
sensitivities.  Analyzer developers can add new analysis techniques
into \safe without too much understanding of the implementation details
of \safe via its extensible APIs.  Finally, \safe provides a debugging
support for analysis developers with visualization and interactive
investigation of the analysis status.  The tool was motivated by the pain
the authors themselves experienced with \oldsafe, which was greatly
relieved by the \htmldebug.  Also, analysis developers can test their
new analysis techniques with extensive Test262, the official ECMAScript
conformance suite.

\bibliographystyle{abbrv}
\bibliography{ref}
\end{document}
